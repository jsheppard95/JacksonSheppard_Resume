%%%%%%%%%%%%%%%%%%%%%%%%%%%%%%%%%%%%%%%%%
% Medium Length Professional CV
% LaTeX Template
% Version 2.0 (8/5/13)
%
% This template has been downloaded from:
% http://www.LaTeXTemplates.com
%
% Original author:
% Rishi Shah 
%
% Important note:
% This template requires the resume.cls file to be in the same directory as the
% .tex file. The resume.cls file provides the resume style used for structuring the
% document.
%
%%%%%%%%%%%%%%%%%%%%%%%%%%%%%%%%%%%%%%%%%

%---------------------------------------------------------------------------
%	PACKAGES AND OTHER DOCUMENT CONFIGURATIONS
%---------------------------------------------------------------------------

\documentclass{resume} % Use the custom resume.cls style

\usepackage[left=0.75in,top=0.6in,right=0.75in,bottom=0.6in]{geometry} % Document margins
\newcommand{\tab}[1]{\hspace{.2667\textwidth}\rlap{#1}}
\newcommand{\itab}[1]{\hspace{0em}\rlap{#1}}
\name{Jackson Sheppard} % Your name
\address{466 Tyndall St, Los Altos, CA, 94022} % Your address
%\address{123 Pleasant Lane \\ City, State 12345} % Your secondary addess (optional)
\address{(650) 862-9401 \\ sheppard@slac.stanford.edu} % Your phone number and email

\begin{document}

%---------------------------------------------------------------------------
%	EDUCATION SECTION
%---------------------------------------------------------------------------

\begin{rSection}{Education}

{\bf University of California, Santa Barbara} \hfill {September 2014 - June 2018} 
\\ Bachelor of Science in Physics \hfill {Received June 16, 2018, GPA: 3.79 } \\ \\
{Relevant Courses:}
\\ Intro to Scientific Computation (Python/Linux), Analog Electronics, Linear Algebra, Quantum Mechanics, Complex Variables, Advanced Mechanics, Electromagnetism, Thermal/Statistical Physics, Fluid Mechanics, Nonlinear Dynamics, Experimental Physics Lab \\ \\
{\bf Georgia Institute of Technology} \hfill {May 7, 2020}
\\ Control of Mobile Robots, an online non-credit course offered through Coursera focusing on dynamics of linear systems and the theory to develop models and formulate stable control systems.
\end{rSection}

%---------------------------------------------------------------------------
%    Honors/Awards
%---------------------------------------------------------------------------

\begin{rSection}{Honors/Awards}
{\bf SLAC: National Accelerator Laboratory}
\\SLAC Spot Award for Dependability \hfill {August 4, 2020 }
\\Selected by SLAC mechanical engineer for implementation of LCLS X-ray optics motion systems.\\ \\
{\bf University of California, Santa Barbara}
\\Dean’s Honors \hfill {Fall 2014, Winter 2015, Spring 2015, Winter 2016, Winter 2017, Fall 2017, Spring 2018 }
\\Department of Physics Academic Honors \hfill {May 13, 2018 }
\end{rSection}

%---------------------------------------------------------------------------
%	QUALIFICATIONS SECTION
%---------------------------------------------------------------------------

\begin{rSection}{Qualifications}

\begin{itemize}
    \item Highly motivated, independent worker, and dedicated to learning new skills.
    \item Good written/oral communication skills and strong organizational skills.
    \item Excellent deductive reasoning/problem solving skills.
    \item Proficient in Python, C++, Linux, Matlab, Epics, GUI Development (edm), PLC Programming (Beckhoff)
\end{itemize}

\end{rSection}

%---------------------------------------------------------------------------
%	WORK EXPERIENCE SECTION
%---------------------------------------------------------------------------

\begin{rSection}{Work Experience}
\begin{rSubsection}{SLAC National Accelerator Laboratory: Menlo Park, CA}{September 2018 - Present}{Science and Engineering Associate, Linac Coherent Light Source (LCLS), Photon Controls and Data Systems (PCDS)}{}
\begin{itemize}
    \item Responsibile for experiment support through integration of user controlled devices into PCDS control system.
    \item Provide on-call technical support for assigned experiments at assigned instruments to troubleshoot common controls problems and escalate when beyond expertise.
    \item Responsible for design, installation, and checkout of LCLS-II motion control systems for X-ray Offset Mirror System (OMS) and Time-resolved atomic, Molecular, and Optical Science instrument (TMO).
\end{itemize}
\end{rSubsection}

\begin{rSubsection}{SLAC National Accelerator Laboratory: Menlo Park, CA}{June 2018 - September 2018}{LCLS Internship Program}{}
\begin{itemize}
    \item Summer student working on beam dynamics of X-Ray Free Electron Laser.
    \item Focusing on efficiency optimization through undulator tapering: varying magnetic field along longitudinal axis to prolong electron energy depletion.
    \item Responsible for characterizing taper profile by developing relationship between magnetic field strength and longitudinal displacement that achieves TW level output power.
\end{itemize}
\end{rSubsection}

\begin{rSubsection}{University of California, Santa Barbra: Goleta, CA}{January 2018 - June 2018}{Undergraduate Research Assistant}{}
\begin{itemize}
    \item Interned in molecular dynamics physical chemistry lab focused on computational techniques of statistical physics to study biological processes.
    \item Worked on convolutional neural network using a variational autoencoder to study peptide folding, model generalizable to other biological problems.
    \item Used encoder to represent ensemble of peptides with one latent space coordinate and characterized the relationship of this coordinate to physical structural patterns.
\end{itemize}
\end{rSubsection}

\begin{rSubsection}{University of California, Santa Barbra: Goleta, CA}{April 2018 - June 2018}{Physics Study Room Fellow}{}
\begin{itemize}
    \item Tutor in “Physics Study Room” at university where undergraduates received help with their physics homework.
    \item Worked alongside graduate students helping students ranging from freshman-level with no background in physics to senior-level taking upper division courses.
\end{itemize}
\end{rSubsection}

\begin{rSubsection}{University of California, Santa Barbra: Goleta, CA}{June 2016 - June 2018}{DSP Proctor}{}
\begin{itemize}
    \item Proctor for Disabled Students Program (DSP), administered exams for students receiving testing accommodations.
\end{itemize}
\end{rSubsection}

\end{rSection}

%---------------------------------------------------------------------------
% Volunteer Experience
%---------------------------------------------------------------------------

\begin{rSection}{Volunteer Experience}
{\bf UCSB Physics Circus: Goletea, CA} \hfill {May 2017 - June 2017}
\begin{itemize}
    \item Program to promote science education at local elementary and high schools in the Santa Barbara area, held physics demonstrations at local “science nights” in Goleta area.
\end{itemize}
\end{rSection}

\end{document}
